\documentclass{resume} 
\usepackage[left=0.75in,top=0.6in,right=0.75in,bottom=0.6in]{geometry}
\usepackage{hyperref} 
\hypersetup{
	colorlinks=true,  
	urlcolor=cyan,
}
\newcommand{\tab}[1]{\hspace{.2667\textwidth}\rlap{#1}}
\newcommand{\itab}[1]{\hspace{0em}\rlap{#1}}
\name{Raidel Nápoles} % Your name
\address{Ingeniero de Software} 
\begin{document}
	
	\begin{rSection}{Contacto}
		\begin{tabular}{ @{} >{\bfseries}l @{\hspace{6ex}} l }
			teléfono & (+53) 53026190\\
			correo & \href{mailto:raidelnapoles95@gmail.com}{raidelnapoles95@gmail.com} \\
			dirección & La Lisa, La Habana, Cuba\\
			linkedin & \href{https://www.linkedin.com/in/raidel-napoles} {linkedin.com/in/raidel-napoles}\\
			github & \href{https://github.com/RaidelNapoles} {RaidelNapoles}
		\end{tabular}
		
	\end{rSection}
	
	
	\begin{rSection}{Experiencia Laboral}
		\begin{rSubsection}{Geocuba}{mayo 2019 - presente}{Programador Backend}{}
			\item Investigación para el cálculo de rutas óptimas en mapas
				\subitem (SqLite, SpatiaLite, Postgres, PostGis, Mapserver, JavaScript)
			\item Programador Backend en los siguientes proyectos: 
				\subitem Sistema de identificación para el censo de población y vivienda 
					\subsubitem (JavaScript, Express, Node.js)
				\subitem Sistema de gestión y control para la actividad científica en las FAR 
					\subsubitem(TypeScript, NestJs, Node.js, PostgreSQL)
		\end{rSubsection}
	
		\begin{rSubsection}{Departamento de Comunicación Institucional del INDER}{ diciembre 2020 - presente}{Programador Fullstack}{}
			\item Programador frontend en el sitio web del repositorio fotográfico del INDER 
				\subitem(HTML, CSS).
			\newline
			\item Programador fullstack en el sitio web de publicaciones perodísticas del INDER 
				\subitem(Python/Django, PostgreSQL, HTML, CSS).
			
		\end{rSubsection}
		
		
	\end{rSection}

	\begin{rSection}{Proyectos Personales - Independientes}
		\begin{rSubsection}{Programador de aplicaciones móviles}{}{}
			\item  Aplicación (apk) InfoPyME
			(\href{https://apklis.cu/application/com.napper.infopyme21}{https://apklis.cu/application/com.napper.infopyme21})
				\subitem (Kotlin, XML, Python, SQL)
		\end{rSubsection}	
	\end{rSection}
	
	\begin{rSection}{Formación}
		
		
		{\bf Universidad de La Habana} \hfill {\em 2014 - 2020} 
		\\ Licenciado en Ciencia de la Computación. \hfill {\em 2020}
		
	\end{rSection}
	
	
	\begin{rSection}{Habilidades}
		
		\begin{tabular}{ @{} >{\bfseries}l @{\hspace{6ex}} l }
			Lenguajes de programación \ & Java, Kotlin, JavaScript, TypeScript, Python, C, C++, HTML5, CSS  \\
			Frameworks & Node.js, Express, NestJs, Django \\
			Bases de datos & SQLite, MySQL, PostgreSQL\\
			Control de versiones  & Git, GitLab, Github\\
			Otras & Habilidad para resolver problemas, Trabajo en equipo
		\end{tabular}
		
	\end{rSection}
	% 
	% 
	\newpage
	
	\begin{rSection}{Idiomas}
		
		\begin{tabular}{ @{} >{\bfseries}l @{\hspace{6ex}} l }
			Nativo & Español\\
			Otros & Inglés (Nivel B2)
		\end{tabular}
		
	\end{rSection}
	
	
	
	
\end{document}
